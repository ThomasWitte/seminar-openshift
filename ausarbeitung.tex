% Based on the IEEE Journal style.
\documentclass[10pt,a4paper,compsoc]{IEEEtran}

\usepackage{graphicx}
\usepackage[cmex10]{amsmath}
% TODO: Deutsche oder Englische Sprache wählen
\usepackage[ngerman]{babel} % Deutsche Ausarbeitung
% \usepackage[USenglish]{babel} % Englische Ausarbeitung
\usepackage{url}
\usepackage{hyperref}
\usepackage[utf8]{inputenc}
% TODO: \usepackage{ccicons} bietet wesentlich hübschere Icons, ist aber nicht auf jedem System verfügbar. Für ccicons muss weiter unten \cc\bysa in \ccbysa geändert werden.
\usepackage{cclicenses}

\begin{document}

% TODO: Titel, Autor und E-Mail Adresse anpassen
\title{Titel}
\author{%
\IEEEauthorblockN{Stefan Dietzel}\\
\IEEEauthorblockA{\url{stefan.dietzel@uni-ulm.de}}%
}

% TODO: Bitte hier Titel des Seminnars und Semester anpassen
\markboth{Titel des Seminars, WS/SS 20XX, Institut für Verteilte Systeme, Universität Ulm}{}
% English: \markboth{Seminar title, WS/SS 20XX, Institut of Distributed Systems, University of Ulm}{}

% TODO: Creative Commons Lizenz: bitte auskommentieren, falls nicht gewünscht
\IEEEpubid{\sf%
\cc\bysa
\quad
Diese Arbeit ist lizensiert unter einer Creative Commons Namensnennung - Weitergabe unter gleichen Bedingungen 3.0 Deutschland Lizenz.}
% English:
% \IEEEpubid{\sf%
% \ccbysa
% \quad
% This work is licensed under a Creative Commons Attribution-ShareAlike 3.0 Germany License.}

\maketitle

\begin{abstract}
Eine kurze Zusammenfassung. Lorem ipsum dolor sit amet, consectetur adipisicing elit, sed do eiusmod tempor incididunt ut labore et dolore magna aliqua. Ut enim ad minim veniam, quis nostrud exercitation ullamco laboris nisi ut aliquip ex ea commodo consequat. Duis aute irure dolor in reprehenderit in voluptate velit esse cillum dolore eu fugiat nulla pariatur. Excepteur sint occaecat cupidatat non proident, sunt in culpa qui officia deserunt mollit anim id est laborum.
\end{abstract}

\section{Introduction}

% TODO: \IEEEPARstart wird nur am Anfang des ersten Absatzes im Paper verwendet, *nicht* am Anfang jeder Section.
\IEEEPARstart{I}n den vergangenen Jahren \dots Lorem ipsum dolor sit amet, consectetur adipisicing elit, sed do eiusmod tempor incididunt ut labore et dolore magna aliqua. Ut enim ad minim veniam, quis nostrud exercitation ullamco laboris nisi ut aliquip ex ea commodo consequat. Duis aute irure dolor in reprehenderit in voluptate velit esse cillum dolore eu fugiat nulla pariatur. Excepteur sint occaecat cupidatat non proident, sunt in culpa qui officia deserunt mollit anim~\cite{test} id est laborum.

Lorem ipsum dolor sit amet, consectetur adipisicing elit, sed do eiusmod tempor incididunt ut labore et dolore magna aliqua. Ut enim ad minim veniam, quis nostrud exercitation ullamco laboris nisi ut aliquip ex ea commodo consequat. Duis aute irure dolor in reprehenderit in voluptate velit esse cillum dolore eu fugiat nulla pariatur. Excepteur sint occaecat cupidatat non proident, sunt in culpa qui officia deserunt mollit anim id est laborum.

Lorem ipsum dolor sit amet, consectetur adipisicing elit, sed do eiusmod tempor incididunt ut labore et dolore magna aliqua. Ut enim ad minim veniam, quis nostrud exercitation ullamco laboris nisi ut aliquip ex ea commodo consequat. Duis aute irure dolor in reprehenderit in voluptate velit esse cillum dolore eu fugiat nulla pariatur. Excepteur sint occaecat cupidatat non proident, sunt in culpa qui officia deserunt mollit anim id est laborum.

\section{Spannendes Thema}

Lorem ipsum dolor sit amet, consectetur adipisicing elit, sed do eiusmod tempor incididunt ut labore et dolore magna aliqua. Ut enim ad minim veniam, quis nostrud exercitation ullamco laboris nisi ut aliquip ex ea commodo consequat. Duis aute irure dolor in reprehenderit in voluptate velit esse cillum dolore eu fugiat nulla pariatur. Excepteur sint occaecat cupidatat non proident, sunt in culpa qui officia deserunt mollit anim id est laborum.

\bibliographystyle{IEEEtranS}
\bibliography{references}

\end{document}
